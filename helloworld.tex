\documentclass[twocolumn]{article}
\usepackage{color}
\usepackage{cite}
\usepackage{draftwatermark}
\usepackage{multirow}
\usepackage{listings}
\usepackage{float}
\usepackage{amsfonts}
\usepackage{amssymb}
\usepackage{amsmath}
\usepackage{amsthm}
\usepackage{epsfig}
\usepackage{epstopdf}
\usepackage{titling}
\usepackage{url}
\usepackage{array}
\usepackage[utf8]{inputenc}
\usepackage[english]{babel}
\SetWatermarkText{DRAFT}
\SetWatermarkScale{5}
\setlength\parskip{.5\baselineskip}
\definecolor{pagecolor}{rgb}{1,0.98,0.9}
\pagecolor{pagecolor}

\author{Ryan J. Kung \\ryankung@ieee.org\\Member, IEEE Blockchain Community }
\title{The Group Theory}

\begin{document}
\maketitle
\section{Group}
\subsection{Definition}
Definition: Group\cite{Cardelli:1996:TS:234313.234418}:

A set $\mathbb{G}={a, b, c, ...}$ is called a group, if tehere exists a group multipliaction connecting the elements in $\mathbb{G}$ in the following way:

(1) $a, b \in \mathbb{G}:\ c=a b \in \mathbb{G}$ (closure)

(2) $a, b, c \in \mathbb{G}: (ab)c=a(bc)$ (associativity)

(3) $\exists e \in \mathbb{G}: ae=e, \forall a \in \mathbb{G}$ (identity / neutral element)

(4) $\forall a \in \mathbb{G}, \exists b \in \mathbb{G}: ab=e, i.e., b\equiv a^{-1}$

\subsection{Pubic-key cryptography}
``The mathematics of public-key cryptography uses a lot of group theory. Different cryptosystems use different groups, such as the group of units in modular arithmetic and the group of rational points on elliptic curves over a finite field. This use of group theory derives not from the "symmetry" perspective, but from the efficiency or difficulty of carrying out certain computations in the groups. Other public-key cryptosystems use other algebraic structures, such as lattices.''' \cite{Cardelli:1996:TS:234313.234418}

\section{Group}
\subsection{Definition}
Definition: Group\cite{Cardelli:1996:TS:234313.234418}:

A set $\mathbb{G}={a, b, c, ...}$ is called a group, if tehere exists a group multipliaction connecting the elements in $\mathbb{G}$ in the following way:

(1) $a, b \in \mathbb{G}:\ c=a b \in \mathbb{G}$ (closure)

(2) $a, b, c \in \mathbb{G}: (ab)c=a(bc)$ (associativity)

(3) $\exists e \in \mathbb{G}: ae=e, \forall a \in \mathbb{G}$ (identity / neutral element)

(4) $\forall a \in \mathbb{G}, \exists b \in \mathbb{G}: ab=e, i.e., b\equiv a^{-1}$

\subsection{Pubic-key cryptography}
``The mathematics of public-key cryptography uses a lot of group theory. Different cryptosystems use different groups, such as the group of units in modular arithmetic and the group of rational points on elliptic curves over a finite field. This use of group theory derives not from the "symmetry" perspective, but from the efficiency or difficulty of carrying out certain computations in the groups. Other public-key cryptosystems use other algebraic structures, such as lattices.''' \cite{Cardelli:1996:TS:234313.234418}
\section{Group}
\subsection{Definition}
Definition: Group\cite{Cardelli:1996:TS:234313.234418}:

A set $\mathbb{G}={a, b, c, ...}$ is called a group, if tehere exists a group multipliaction connecting the elements in $\mathbb{G}$ in the following way:

(1) $a, b \in \mathbb{G}:\ c=a b \in \mathbb{G}$ (closure)

(2) $a, b, c \in \mathbb{G}: (ab)c=a(bc)$ (associativity)

(3) $\exists e \in \mathbb{G}: ae=e, \forall a \in \mathbb{G}$ (identity / neutral element)

(4) $\forall a \in \mathbb{G}, \exists b \in \mathbb{G}: ab=e, i.e., b\equiv a^{-1}$

\subsection{Pubic-key cryptography}
``The mathematics of public-key cryptography uses a lot of group theory. Different cryptosystems use different groups, such as the group of units in modular arithmetic and the group of rational points on elliptic curves over a finite field. This use of group theory derives not from the "symmetry" perspective, but from the efficiency or difficulty of carrying out certain computations in the groups. Other public-key cryptosystems use other algebraic structures, such as lattices.''' \cite{Cardelli:1996:TS:234313.234418}
\section{Group}
\subsection{Definition}
Definition: Group\cite{Cardelli:1996:TS:234313.234418}:

A set $\mathbb{G}={a, b, c, ...}$ is called a group, if tehere exists a group multipliaction connecting the elements in $\mathbb{G}$ in the following way:

(1) $a, b \in \mathbb{G}:\ c=a b \in \mathbb{G}$ (closure)

(2) $a, b, c \in \mathbb{G}: (ab)c=a(bc)$ (associativity)

(3) $\exists e \in \mathbb{G}: ae=e, \forall a \in \mathbb{G}$ (identity / neutral element)

(4) $\forall a \in \mathbb{G}, \exists b \in \mathbb{G}: ab=e, i.e., b\equiv a^{-1}$

\subsection{Pubic-key cryptography}
``The mathematics of public-key cryptography uses a lot of group theory. Different cryptosystems use different groups, such as the group of units in modular arithmetic and the group of rational points on elliptic curves over a finite field. This use of group theory derives not from the "symmetry" perspective, but from the efficiency or difficulty of carrying out certain computations in the groups. Other public-key cryptosystems use other algebraic structures, such as lattices.''' \cite{Cardelli:1996:TS:234313.234418}
\section{Group}
\subsection{Definition}
Definition: Group\cite{Cardelli:1996:TS:234313.234418}:

A set $\mathbb{G}={a, b, c, ...}$ is called a group, if tehere exists a group multipliaction connecting the elements in $\mathbb{G}$ in the following way:

(1) $a, b \in \mathbb{G}:\ c=a b \in \mathbb{G}$ (closure)

(2) $a, b, c \in \mathbb{G}: (ab)c=a(bc)$ (associativity)

(3) $\exists e \in \mathbb{G}: ae=e, \forall a \in \mathbb{G}$ (identity / neutral element)

(4) $\forall a \in \mathbb{G}, \exists b \in \mathbb{G}: ab=e, i.e., b\equiv a^{-1}$

\subsection{Pubic-key cryptography}
``The mathematics of public-key cryptography uses a lot of group theory. Different cryptosystems use different groups, such as the group of units in modular arithmetic and the group of rational points on elliptic curves over a finite field. This use of group theory derives not from the "symmetry" perspective, but from the efficiency or difficulty of carrying out certain computations in the groups. Other public-key cryptosystems use other algebraic structures, such as lattices.''' \cite{Cardelli:1996:TS:234313.234418}
\section{Group}
\subsection{Definition}
Definition: Group\cite{Cardelli:1996:TS:234313.234418}:

A set $\mathbb{G}={a, b, c, ...}$ is called a group, if tehere exists a group multipliaction connecting the elements in $\mathbb{G}$ in the following way:

(1) $a, b \in \mathbb{G}:\ c=a b \in \mathbb{G}$ (closure)

(2) $a, b, c \in \mathbb{G}: (ab)c=a(bc)$ (associativity)

(3) $\exists e \in \mathbb{G}: ae=e, \forall a \in \mathbb{G}$ (identity / neutral element)

(4) $\forall a \in \mathbb{G}, \exists b \in \mathbb{G}: ab=e, i.e., b\equiv a^{-1}$

\subsection{Pubic-key cryptography}
``The mathematics of public-key cryptography uses a lot of group theory. Different cryptosystems use different groups, such as the group of units in modular arithmetic and the group of rational points on elliptic curves over a finite field. This use of group theory derives not from the "symmetry" perspective, but from the efficiency or difficulty of carrying out certain computations in the groups. Other public-key cryptosystems use other algebraic structures, such as lattices.''' \cite{Cardelli:1996:TS:234313.234418}
\section{Group}
\subsection{Definition}
Definition: Group\cite{Cardelli:1996:TS:234313.234418}:

A set $\mathbb{G}={a, b, c, ...}$ is called a group, if tehere exists a group multipliaction connecting the elements in $\mathbb{G}$ in the following way:

(1) $a, b \in \mathbb{G}:\ c=a b \in \mathbb{G}$ (closure)

(2) $a, b, c \in \mathbb{G}: (ab)c=a(bc)$ (associativity)

(3) $\exists e \in \mathbb{G}: ae=e, \forall a \in \mathbb{G}$ (identity / neutral element)

(4) $\forall a \in \mathbb{G}, \exists b \in \mathbb{G}: ab=e, i.e., b\equiv a^{-1}$

\subsection{Pubic-key cryptography}
``The mathematics of public-key cryptography uses a lot of group theory. Different cryptosystems use different groups, such as the group of units in modular arithmetic and the group of rational points on elliptic curves over a finite field. This use of group theory derives not from the "symmetry" perspective, but from the efficiency or difficulty of carrying out certain computations in the groups. Other public-key cryptosystems use other algebraic structures, such as lattices.''' \cite{Cardelli:1996:TS:234313.234418}
\section{Group}
\subsection{Definition}
Definition: Group\cite{Cardelli:1996:TS:234313.234418}:

A set $\mathbb{G}={a, b, c, ...}$ is called a group, if tehere exists a group multipliaction connecting the elements in $\mathbb{G}$ in the following way:

(1) $a, b \in \mathbb{G}:\ c=a b \in \mathbb{G}$ (closure)

(2) $a, b, c \in \mathbb{G}: (ab)c=a(bc)$ (associativity)

(3) $\exists e \in \mathbb{G}: ae=e, \forall a \in \mathbb{G}$ (identity / neutral element)

(4) $\forall a \in \mathbb{G}, \exists b \in \mathbb{G}: ab=e, i.e., b\equiv a^{-1}$

\subsection{Pubic-key cryptography}
``The mathematics of public-key cryptography uses a lot of group theory. Different cryptosystems use different groups, such as the group of units in modular arithmetic and the group of rational points on elliptic curves over a finite field. This use of group theory derives not from the "symmetry" perspective, but from the efficiency or difficulty of carrying out certain computations in the groups. Other public-key cryptosystems use other algebraic structures, such as lattices.''' \cite{Cardelli:1996:TS:234313.234418}

\bibliographystyle{unsrt}
\bibliography{./cites}
\end{document}